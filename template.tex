%%%%%%%%%%%%%%%%%%%%%%%%%%%%%%%%%%%%%%%%%
% Twenty Seconds Resume/CV
% LaTeX Template
% Version 1.1 (8/1/17)
%
% This template has been downloaded from:
% http://www.LaTeXTemplates.com
%
% Original author:
% Carmine Spagnuolo (cspagnuolo@unisa.it) with major modifications by 
% Vel (vel@LaTeXTemplates.com)
%
% License:
% The MIT License (see included LICENSE file)
%
%%%%%%%%%%%%%%%%%%%%%%%%%%%%%%%%%%%%%%%%%

%----------------------------------------------------------------------------------------
%	PACKAGES AND OTHER DOCUMENT CONFIGURATIONS
%----------------------------------------------------------------------------------------

\documentclass[letterpaper]{twentysecondcv} % a4paper for A4

%----------------------------------------------------------------------------------------
%	 PERSONAL INFORMATION
%----------------------------------------------------------------------------------------

% If you don't need one or more of the below, just remove the content leaving the command, e.g. \cvnumberphone{}



\cvname{Bhargav N} % Your name
\cvjobtitle{Software Engineer} % Job title/career

\cvdate{11 April 1996} % Date of birth
\cvaddress{\#401, 71/2, Ichangur, Attibele, Anekal, Bengaluru-562107} % Short address/location, use \newline if more than 1 line is required
\cvnumberphone{+91 8105406750} % Phone number
\cvsite{https://github.com/bhargav516} % Personal website
\cvmail{bhargav19968@gmail.com} % Email address

%----------------------------------------------------------------------------------------

\begin{document}
%----------------------------------------------------------------------------------------
%	 Education
%----------------------------------------------------------------------------------------

\Education{B.E Computer Science \newline PES South Campus | 2018 | 70.49 \%
\newline \newline Class XII \newline  krupanidhi College | 2014 | 83.6 \%
\newline \newline Class X \newline  swamy vivekananda vidyaniketana neralur | 2012 | 89.28 \%} % To have no Education section, just remove all the text and leave \Education{}

%----------------------------------------------------------------------------------------
%	 SKILLS
%----------------------------------------------------------------------------------------

% Skill bar section, each skill must have a value between 0 an 6 (float)
\skills{Languages: C, Java, Python
\newline  Web : HTML, CSS, JS, XML
\newline  General : Object Oriented Programming, AI/ML, Big data
\newline  Other: REST, spring boot, Micro-services, RabbitMQ, SQL/No SQL, Kafka streaming, elastic search, Debezium, Fluentd, Hibernate, docker, Kubernetes, Gradle, Hazelcast cache
\newline Operating systems : Windows, Linux}



\ExtraCurricular{\newline\\ - Received Bronze Star Award at Honeywell for passion and innovation

%\\ - Received Best Debut Award at Honeywell.%
\\ - Received Bravo Star Award at Honeywell for Quick learner and executor
}
\Achievements{\newline \\ - Won 2 Prize at Bit hackathon(2015) for "GreenBen" Application.
\\ - Topper in math and social studies at school level}

\Interest{\newline\\ - Gaming and interested in new technologies and innovation. 
\\ - Fond of Traveling and Trekking, food enthusiast.}
%------------------------------------------------

% Skill text section, each skill must have a value between 0 an 6


%----------------------------------------------------------------------------------------

\makeprofile % Print the sidebar

%----------------------------------------------------------------------------------------
%	 Experience
%----------------------------------------------------------------------------------------

\section{Work Experience and Internships}

\begin{twenty} % Environment for a list with descriptions
	\twentyitem{ Since july'18}{Software Engineer 2}{Honeywell, Bengaluru}{ {\large Intelligrated} (Warehouse Execution Systems) \\
	  \\- Developing Automated warehouse solutions for large scale warehouse using Spring Boot and cloud, Micro-services, SQL/NO SQL databases, AMQP/kafka, Elastic Search, Fluentd, Hazelcast for Remote Caching and DROOLS for Rule Engine. 
	  \\- Worked under Fellow Engineer and developed Machine learning models, which improved the movement of containers inside the Warehouse.\\- Created a graphical Simulator using OpenGL and c++, turned it into product \\- Worked along with data team to create a data streaming framework using Kafka streams, framework was built for spring boot micro-services application. \\- Worked along with platform team on creating framework that generates micro services applications.\\- Involved in requirement analysis, architectural design, estimations ,implementation, Review and testing.\\}
	\twentyitem{June'17}{Summer Intern}{Vantage Agora, Bengaluru}{- Learnt about Bussiness operating system \& Analyzed workflow \\- suggested optimization of few bottlenecks}
	%\twentyitem{<dates>}{<title>}{<location>}{<description>}
\end{twenty}


%----------------------------------------------------------------------------------------
%	 PROJECTS
%----------------------------------------------------------------------------------------

\section{Projects}

\begin{twenty} % Environment for a short list with no descriptions
	\twentyitem{Jan-Apr'18}{ Inference of Crime Using Big Data} {PES Final year project}{Studied the crime rate of Chicago, reduce crime rate giving the authorities to find ways to curb crime. The crime rate is projected by incorporating Nodal (Demographics and Point of Interest data) and Edge (Taxi Flow and Geographic data) features into a regression model.}
	\twentyitem{Jun-Oct'17}{ PayPes}{PES}{This was a drive to make cashless transaction possible at college using id cards. The digital Wallet was built using Firebase services, which used NOSQL database. An android application was built using android studio and included functionalists such as scan bar code, transactions etc, Where both merchants and Customers had different pages. The application was backed by Razorpay for making payments.}
	%\twentyitem{<dates>}{<title>}{<location>}{<description>}
\end{twenty}

%----------------------------------------------------------------------------------------
%	 ELECTIVES AND Certificate 
%----------------------------------------------------------------------------------------

\section{Electives and MOOCs}

\begin{twentyshort} % Environment for a list with descriptions

	\twentyitemshort{Electives}{Design Patterns, Pattern Recognition, Artificial Intelligence, Architectural patterns }
	\twentyitemshort{MOOCs}{\textbf{- Master SQL \& Python Courses on Datacamp} \\- Certified SAFe 4 Practitioner from Scaled Agile}
	%\twentyitemshort{<dates>}{<title/description>}
\end{twentyshort}

%----------------------------------------------------------------------------------------
%	 OTHER INFORMATION
%----------------------------------------------------------------------------------------




%----------------------------------------------------------------------------------------
%	 SECOND PAGE EXAMPLE
%----------------------------------------------------------------------------------------

%\newpage % Start a new page

%\makeprofile % Print the sidebar

%\section{Other information}

%\subsection{Review}

%Alice approaches Wonderland as an anthropologist, but maintains a strong sense of noblesse oblige that comes with her class status. She has confidence in her social position, education, and the Victorian virtue of good manners. Alice has a feeling of entitlement, particularly when comparing herself to Mabel, whom she declares has a ``poky little house," and no toys. Additionally, she flaunts her limited information base with anyone who will listen and becomes increasingly obsessed with the importance of good manners as she deals with the rude creatures of Wonderland. Alice maintains a superior attitude and behaves with solicitous indulgence toward those she believes are less privileged.

%\section{Other information}

%\subsection{Review}

%Alice approaches Wonderland as an anthropologist, but maintains a strong sense of noblesse oblige that comes with her class status. She has confidence in her social position, education, and the Victorian virtue of good manners. Alice has a feeling of entitlement, particularly when comparing herself to Mabel, whom she declares has a ``poky little house," and no toys. Additionally, she flaunts her limited information base with anyone who will listen and becomes increasingly obsessed with the importance of good manners as she deals with the rude creatures of Wonderland. Alice maintains a superior attitude and behaves with solicitous indulgence toward those she believes are less privileged.

%----------------------------------------------------------------------------------------

\end{document} 
